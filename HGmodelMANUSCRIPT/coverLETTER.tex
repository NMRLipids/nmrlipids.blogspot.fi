\documentclass[12pt]{letter}
\usepackage{graphicx,baskervald,microtype}
\usepackage{hyperref,amsmath,xcolor}
\usepackage{pgfplots,marginnote}
\usepackage[top=0.5in,bottom=0.5in,left=1in,right=1in]{geometry}
%\newgeometry{margin=2.5cm}
\newcommand{\omamargin}[1]{\marginnote{\textbf{#1}}[7pt]}
\begin{document}
\reversemarginpar
\pagestyle{empty}
\noindent O. H. Samuli Ollila \\
\noindent Helsinki Biophysics and Biomembrane Group\\
\noindent Department of Biomedical Engineering and Computational Science\\
\noindent P.O. Box 12200, FI-00076 Aalto University, Espoo, Finland\\
\noindent samuli.ollila@aalto.fi, Tel. +358503746963, Fax. +358 9 470 23182 \\


%THESE ARE COPYPASTED FROM THE JACS AUTHOR GUIDELINES AND I HAVE TRIED TO ANWER THESE
%
%Articles of high scientific quality, originality, significance, and conceptual novelty that
%are of interest to the wide and diverse contemporary readership
%of JACS will be given priority for publication.
%
%it is required that (a) the level of theory and methodology employed must be adequate for the
%problem at hand, and (b) the manuscript must provide significant chemical insights or have
%substantial predictive value.

To Whom It May Concern,

Please find attached to this message a manuscript that we would like to submit for publication 
in \textit{Journal of the american chemical society} (JACS): 
`Towards atomistic resolution structure of phosphatidylcholine glycerol backbone and choline headgroup at different ambient conditions', 
by A. Botan et al. In the manuscript we combine extensive amount of experimental Nuclear Magnetic Resonance (NMR)
and molecular dynamics simulation data to understand the atomistic resolution structure of biologically abundant 
phospholipid molecules and their assemblies.

We believe that the manuscript is interesting for the wide and diverse contemporary audience of
JACS since the used combination of experiments and simulations is shown to be successfull in
resolving the atomistic resolution structure of biomolecules in various biologically relevant conditions.
The approach is demonstrated for lipids, however, the extension to, e.g. membrane proteins is straightforward. 
Such an atomistic resolution of  biomolecular structure
significantly advances the understanding in fundamental chemistry, molecular biology, and also supports the development of applications, 
for example new biomaterials or drug delivery systems.

The work has been conducted by using a novel open collaboration concept, and all the scientific contributions
are done publicly through the nmrlipids.blogspot.fi. Thus all the scientific content related to this work
(including raw data, discussions, manuscript drafts, etc.)
has been publicly available during the whole course of the project through the blog. This material and discussion possibility
will be kept available also after the publication. This approach has significantly increased and will increase the scientific
quality and impact of the work. In addition, we have shared all the simulation raw data in such a format
that it can be reanalyzed easily also for other purposes. In addition to the scientic content,
we believe that the success of this approach in the field of chemistry is highly interesting 
for the JACS audience.

The usage of molecular dynamics simulations and open collaboration are adequate for the
problem at hand. Indeed, such simulations are currently the most straightforward method to 
construct atomistic resolution structures to interpret NMR experiments, but
enormous amount of simulation work was required for this work. The only practical
solution to collect this data is the open collaboration.
As result, we have produced original chemical insight for the atomistic resolution 
structures of phospholipids in various biologically relevant conditions.

We hope that you can consider this manuscript to be published as an article in the Journal of the american chemical society. \\




Sincerely yours,

On behalf of the authors,

O. H. Samuli Ollila



\end{document}
